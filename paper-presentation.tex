% Files should be in encoding corresponding to the setting of \usepackage[...]{inputenc}

\documentclass[%      Basic settings of the document
%  draft,    				  % Turn on draft compilation
  14pt,       				% 12 points default font size
	c,                  % The slides should always start at their top (not vertically centered)
	aspectratio=1610,   % Ratio of width and height is 16:10 (corresponding to lecture rooms at Technická 12),
	                    % other options are 43, 149, 169, 54, 32.
	unicode,						% Bookmarks and metainformation in the compiled PDF will be unicode
]{beamer}				    	% Document type 'beamer' is the most often used one for slides
%\usepackage{etex}

\usepackage{hyperref}
\usepackage[utf8]		  %	The input files are in UTF-8 (unicode)
	{inputenc}					% Package for setting input encoding
	
\usepackage{graphicx} % Package for inclusion of external graphics

\usepackage[
	nohyperlinks				% No hyperlinks to the list of acronyms will be created
]{acronym}						% Package 'acronym' for automated handling of abbreviations and symbols
											% Neccessary for the 'acronym' environment of 'thesis' to work

\usepackage{cmap} 		% This package allows the final PDF
											% is fully "searchable" and "copyable"

%\usepackage{amsmath} %for typesetting advanced mathematics

\usepackage{booktabs} % Package allowing the use of commands \toprule, \midrule, \bottomrule
                      % when creating tables


%%%%%%%%%%%%%%%%%%%%%%%%%%%%%%%%%%%%%%%%%%%%%%%%%%%%%%%%%%%%%%%%%
%%%%%%%%                   Definitions                 %%%%%%%%%%
%%%%%%%%%%%%%%%%%%%%%%%%%%%%%%%%%%%%%%%%%%%%%%%%%%%%%%%%%%%%%%%%%

\input{settings}      % please go into that file and fill information about you, your topic etc.


%%%%%%%%%%%%%%%%%%%%%%%%%%%%%%%%%%%%%%%%%%%%%%%%%%%%%%%%%%%%%%%%%%%%%%%%
%%%%%   Setting values that appear in the Properties of the PDF  %%%%%%%
%%%%%%%%%%%%%%%%%%%%%%%%%%%%%%%%%%%%%%%%%%%%%%%%%%%%%%%%%%%%%%%%%%%%%%%%
%% If 'hyperref' package is loaded then command '\pdfsettings' can be simply used
\pdfsettings
%  However, you can set the individual fields by hand:
%\hypersetup{
%  pdftitle={Title of the thesis},    	% 'Document Title' field
%  pdfauthor={Author name},   	        % 'Author' field
%  pdfsubject={Type of the document}, 	% 'Subject' field
%  pdfkeywords={Keywords}           	  % 'Keywords' field
%}
\hypersetup{pdfpagemode=FullScreen}  % when opening the file with Adobe Reader,
                                     % automatically open it in fullscreen mode
%%%%%%%%%%%%%%%%%%%%%%%%%%%%%%%%%%%%%%%%%%%%%%%%%%%%%%%%%%%%%%%%%%%%%%%

\usetheme{VUT} 				% colours and layout corresponding to the BUT style
% alternatively, other styles can be used, such as
%\usetheme{Darmstadt} \usecolortheme{default2}
% but with no guarantees
\logoheader					% inserts small BUT FEEC logotype in the header; leave this in use




\begin{document}

% if you do not use the following command, the lower right corner of each slide will contain active navigation symbols
\disablenavigationsymbols

% the title slide will be created without header, footer etc.
\maketitle

\begin{frame}
  \frametitle{Table of contents}
  \large{
    \begin{itemize}
      \item Introduction
      \item Article's contribution
      \item Proposed solution
      \item Results
      \item Conclusion
    \end{itemize}
  }
\end{frame}

\begin{frame}
  \frametitle{Ring signatures}
  \large{
    \begin{columns}[T]
      \begin{column}{0.8\textwidth}
        \begin{itemize}
          \item Asymmetric cryptography
          \item Derived from group signatures
          \item Sign a message on behalf of a group of signers
          \item Preserves the anonymity of the signer
          \item Used in \textbf{cryptocurrencies}, \textbf{e-voting}
        \end{itemize}
      \end{column}
      \begin{column}{0.2\textwidth}
        \begin{figure}[htbp]
          \centering
          \includegraphics[width=\textwidth]{images/monero.png}
        \end{figure}
      \end{column}
    \end{columns}
  }
\end{frame}

\begin{frame} 
  \frametitle{Linkability of ring signatures}
  \large{
    \begin{itemize}
      \item Known as Linkable Ring Signatures
      \item A signer generates a label (tag)
      \item A label is generated from the private key
      \item Detect whether two signatures were generated by the \textbf{same signer}
      \item Used in Monero to prevent double spending
    \end{itemize}
  }
\end{frame}

\begin{frame} 
  \frametitle{Post-quantum cryptography}
  \large{
    \begin{itemize}
      \item Currently used algorithms can be broken using quantum computers
      \item Applies mostly to asymmetric cryptography
      \item Best quantum computer has \textbf{127 physical qubits} (IBM)[1]
      \item \textbf{Tens of millions} bits required to efficiently break asymmetric cryptography
    \end{itemize}
  }
\end{frame}

\begin{frame}
  \frametitle{Proposed solution}
  \large{
    \begin{itemize}
      \item A need to create post-quantum ring signatures
      \item Authors introduce \textbf{Lattice-Based one-time Linkable Ring signature} (L2RS) scheme
      \item Cryptocurrency privacy-preserving protocol called \textbf{Lattice RingCT v1.0}
      \item Monero currently uses Ring CT
    \end{itemize}
  }
\end{frame}

\begin{frame}
  \frametitle{L2RS properties}
  \large{
    \begin{itemize}
			\item Resistant to quantum computers attacks
      \item Based on Bimodal Lattice Signature Scheme (BLISS)
      \item BLISS is an ancestor of Dilithium
      \item Relies on the \textbf{Ring-SIS} (Short Integer Solution) problem
    \end{itemize}
  }
\end{frame}

\begin{frame}
  \frametitle{L2RS properties}
  \large{
    \begin{itemize}
      \item \textbf{Unforgeability} -- An adversary can't forge a signature
      \item \textbf{Linkability} -- Able to deceted wheter two signatures were created using the sk
      \item \textbf{Non-slanderability} -- An adversary can't prove that two signatures are linked if they were signed with different sk's
      \item \textbf{Unconditional anonymity} -- An adversary can't distinguish a signer's pk with unlimited computing resources
    \end{itemize}
  }
\end{frame}


\begin{frame}
  \frametitle{L2RS foundations}
  \large{
    \begin{itemize}
      \item The main structure is a \textbf{ring} $R_q = \mathbb{Z}_q[X]/f(x)$
      \item The number $q$ is a prime number where $q=1\,\mathrm{mod}\,2n$ and $n=512$
      \item \textbf{Private key} consists of $m-1$ rings (polynomials) where $f(x) = x^n - 1$ and coefficients modulo $q$ 
      \item \textbf{The public key} consists of only one ring
      \item \textbf{2 Public parameters} are the same size as the private key
    \end{itemize}
  }
\end{frame}

\begin{frame}
  \frametitle{Rings}
  \large{
    An example of a private key \textbf{S} for $m=6$, $n=512$
    \begin{itemize}
      \item $s_1 = s_{1,0} + s_{1,1}x + s_{1,2}x^2 + \dots + s_{1,510}x^{510} + s_{1,511}x^{511}$
      \item $s_2 = s_{2,0} + s_{2,1}x + s_{2,2}x^2 + \dots + s_{2,510}x^{510} + s_{2,511}x^{511}$
      \item $s_3 = s_{3,0} + s_{3,1}x + s_{3,2}x^2 + \dots + s_{3,510}x^{510} + s_{3,511}x^{511}$
      \item $s_4 = s_{4,0} + s_{4,1}x + s_{4,2}x^2 + \dots + s_{4,510}x^{510} + s_{4,511}x^{511}$
      \item $s_5 = s_{5,0} + s_{5,1}x + s_{5,2}x^2 + \dots + s_{5,510}x^{510} + s_{5,511}x^{511}$
    \end{itemize}
  }
\end{frame}

\begin{frame}
	\frametitle{L2RS algorithms}
	\large{
		\begin{itemize}
			\item \textbf{\texttt{L2RS.Setup}} -- Produces the public parameters
			\item \textbf{\texttt{L2RS.KeyGen}} -- Generate the key pair from the public parameter, generated \textit{uniformly and independently}
			\item \textbf{\texttt{L2RS.SigGen}} -- Generate a signature from a private key, public keys and public parameters
			\item \textbf{\texttt{L2RS.SigVer}} -- Verifies the signature using the public keys and public parameters
			\item \textbf{\texttt{L2RS.SigLink}} -- Takes two signatures and checks whether they are linked
		\end{itemize}
	}
\end{frame}

\begin{frame}
  \frametitle{L2RS signing/verifying}
  \large{
    \begin{itemize}
      \item $w$ number of participants
      \item All the public \textbf{keys are colleted} including the signers
      \item The signatures consists of a hash $c_1$ and rings $t_1,\dots,t_w$
      \item It is created using all the \textbf{public keys} the \textbf{signers private key} and \textbf{the public parameters}
      \item The signature can verified using the \textbf{public keys} and \textbf{public parameters} while keeping the signers identity
    \end{itemize}
  }
\end{frame}

\begin{frame}
  \frametitle{RingCT v1.0}
  \large{
    \begin{itemize}
      \item Utilizes a \textbf{homomorphic commitment} as an additional primitive
      \item A party can commit to a chosen value while keeping it a secret
      \item The value can be revealed later
      \item Algorithms for this protocol are also defined: \textbf{\texttt{LRCT.Mint}} and \textbf{\texttt{LRCT.Spend}}
    \end{itemize}
  }
\end{frame}

\begin{frame}
  \frametitle{Results}
  \large{
    \begin{itemize}
      \item Many of the security claims are well described in the papers appendixes
      \item This scheme was used to build an extension of the original Ring CT protocol used by Monero cryptocurrency
      \item There is no reference implementation by the authors so there are no to performance results
    \end{itemize}
  }
\end{frame}

\begin{frame}
  \frametitle{Conclusion}
  \large{
    \begin{itemize}
      \item The authors successfully managed to create a secure post quantum ring signature scheme
      \item They propose a real use case in cryptocurrencies
      \item The L2RS scheme can be used in many areas not only cryptocurrencies as showcased in this paper
    \end{itemize}
  }
\end{frame}

\begin{frame}
  \frametitle{Bibliography}
  \small{
    [1] IBM Unveils Breakthrough 127-Qubit Quantum Processor. IBM. \textit{IBM Newsroom} [online]. 2021 [cit. 2022-10-26]. Available from: \url{https://shorturl.at/vDJT4}
  }
\end{frame}

\begin{frame}[c]
  \frametitle{\mbox{ }}
  \begin{center}
    {\Huge Thank you for your attention!}
  \end{center}
\end{frame}

\end{document}
